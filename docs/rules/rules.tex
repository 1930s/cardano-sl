% -*- latex -*-

\documentclass{article}

%% packages %%

\usepackage{amsmath}
\usepackage{amssymb}
\usepackage{mathtools}
\usepackage{enumitem}
\usepackage{unicode-math}
\usepackage{newunicodechar}
\usepackage[dvipsnames]{xcolor}
\usepackage[tikz]{bclogo}

\makeatletter
\AtBeginDocument{
  \usepackage
  [ raiselinks=false
  , breaklinks=true
  , colorlinks=true
  , citecolor=blue
  , urlcolor=blue
  , linkcolor=PineGreen
  %
  , pdfpagemode=UseOutlines
  , bookmarksopen=true
  %
  , pdftitle={\@title}
  , pdfauthor={\@author}]
  {hyperref}
}
\makeatother

%% /packages %%

%% Marginal notes %%

\usepackage{xargs}
\usepackage[colorinlistoftodos,prependcaption,textsize=tiny]{todonotes}
\newcommandx{\unsure}[2][1=]{\todo[linecolor=red,backgroundcolor=red!25,bordercolor=red,#1]{#2}}
\newcommandx{\info}[2][1=]{\todo[linecolor=green,backgroundcolor=green!25,bordercolor=green,#1]{#2}}
\newcommandx{\change}[2][1=]{\todo[linecolor=blue,backgroundcolor=blue!25,bordercolor=blue,#1]{#2}}
\newcommandx{\inconsistent}[2][1=]{\todo[linecolor=blue,backgroundcolor=blue!25,bordercolor=red,#1]{#2}}
\newcommandx{\critical}[2][1=]{\todo[linecolor=blue,backgroundcolor=blue!25,bordercolor=red,#1]{#2}}
\newcommand{\improvement}[1]{\todo[linecolor=pink,backgroundcolor=pink!25,bordercolor=pink]{#1}}
\newcommandx{\resolved}[2][1=]{\todo[linecolor=OliveGreen,backgroundcolor=OliveGreen!25,bordercolor=OliveGreen,#1]{#2}} % use this to mark a resolved question

%% /Marginal notes %%

%% macros %%

\newenvironment{record}{\begin{description}[font=\it]}{\end{description}}
\newenvironment{currentvalue}{\begin{bclogo}[noborder=true,
    logo=\bcinfo, epBarre=3, couleurBarre=YellowOrange]{Value in Cardano}}{\end{bclogo}}
\newenvironment{implementationhint}{\begin{bclogo}[noborder=true,
    logo=\bccrayon, epBarre=3, couleurBarre=PineGreen]{Implementation hint}}{\end{bclogo}}
\newenvironment{treacherous}{\begin{bclogo}[noborder=true,
    logo=\bcdz, epBarre=3, couleurBarre=Red]{Beware}}{\end{bclogo}}

\newcommand{\powerset}[1]{\mathcal{P}(#1)}
\newunicodechar{⧵}{\ensuremath{\smallsetminus}}

\newcommand{\lists}[1]{{#1}^*}
\newcommand{\nelists}[1]{{#1}^+}
\newcommand{\emptylist}{ε}

\newcommand{\pos}[2]{{#2}_{#1}}
\newcommand{\length}[1]{|#1|}
\newcommand{\rem}[2]{\mathrm{rem}_{#1}(#2)}
\newcommand{\supportsize}[1]{|#1|}
\newcommand{\extension}[2]{{#1}^*{#2}}

\newcommand{\kronecker}[2]{[#1 = #2]}

\newcommand{\idsof}[1]{\mathcal{I}\!_#1}
\newcommand{\id}{\mathsf{id}}
\newcommand{\txs}{\mathcal{T}}
\newcommand{\txids}{\idsof{\txs}}
\newcommand{\txinit}{\txs_{\!\mathsf{init}}}
\newcommand{\blocks}{\mathcal{B}}
\newcommand{\rblocks}{\mathcal{B}_r}
\newcommand{\bblocks}{\mathcal{B}_b}
\newcommand{\blockids}{\idsof{\blocks}}
\newcommand{\agentids}{\idsof{\mathcal{A}}}
\newcommand{\bootstrapstakeholders}{\mathcal{A}_{\mathsf{bootstrap}}}
\newcommand{\slotleader}[3]{\mathsf{sl}(#1,#2,#3)}
\newcommand{\recipient}[1]{\mathsf{recipient}(#1)}
\newcommand{\amount}[1]{\mathsf{amount}(#1)}
\newcommand{\transactions}[1]{\mathsf{transactions}(#1)}
\newcommand{\inputs}[1]{\mathsf{inputs}(#1)}
\newcommand{\outputs}[1]{\mathsf{outputs}(#1)}
\newcommand{\intrans}[1]{\mathsf{trans}(#1)}
\newcommand{\inindex}[1]{\mathsf{index}(#1)}
\newcommand{\witnesses}[1]{\mathsf{witnesses}(#1)}
\newcommand{\minimumfee}[1]{\mathsf{minfee}(#1)}
\newcommand{\stakedistributions}{\mathcal{D}}
\newcommand{\seeds}{\mathcal{S}}
\newcommand{\initialdistribution}{d_{\mathsf{init}}}
\newcommand{\initialseed}{s_{\mathsf{init}}}
\newcommand{\initialdelegation}{g_{\mathsf{init}}}
\newcommand{\transitivefix}[1]{{#1}^ω}
\newcommand{\delegations}[1]{\mathsf{deleg}(#1)}
\newcommand{\epoch}[1]{\mathsf{epoch}(#1)}
\newcommand{\epochslot}[1]{\mathsf{slot}_\mathsf{epoch}(#1)}
\newcommand{\epochindex}[1]{\mathsf{epoch}_\mathsf{index}(#1)}
\newcommand{\epochseed}[1]{\mathsf{seed}(#1)}
\newcommand{\epochdistribution}[1]{\mathsf{distr}(#1)}
\newcommand{\currentdistribution}[1]{\mathsf{distr}_0(#1)}
\newcommand{\blocktrees}{\mathcal{G}}
\newcommand{\blockchains}{\mathcal{D}}
\newcommand{\maxchains}[1]{\mathcal{M}(#1)}
\newcommand{\height}[1]{\mathsf{height}(#1)}
\newcommand{\delegator}[1]{\mathsf{delegator}(#1)}
\newcommand{\delegatee}[1]{\mathsf{delegatee}(#1)}
\newcommand{\delegationcertificates}[1]{\mathsf{certificates}_\mathsf{delegation}(#1)}
\newcommand{\selectrichman}[1]{\mathsf{select}_\mathsf{bootstrap}(#1)}
\newcommand{\stakerepartitions}{\mathcal{R}}
\newcommand{\repartition}[1]{\mathsf{repartition}(#1)}
\newcommand{\bootstraprepartition}{r_\mathsf{bootstrap}}
\newcommand{\stakeof}[2]{\mathsf{stake}(#1,#2)}
\newcommand{\maxmempoolsize}{\mathsf{max}_{\mathsf{poolsize}}}
\newcommand{\quality}[1]{\mathsf{quality}(#1)}
\newcommand{\commitphase}{\mathsf{phase}_\mathsf{commitment}}
\newcommand{\openphase}{\mathsf{phase}_\mathsf{open}}
\newcommand{\recoveryphase}{\mathsf{phase}_\mathsf{recovery}}
\newcommand{\epochcommitments}[1]{\mathsf{commitments}(#1)}
\newcommand{\epochopens}[1]{\mathsf{opens}(#1)}
\newcommand{\epochrecoveries}[1]{\mathsf{recoveries}(#1)}
\newcommand{\richmen}[2]{\mathsf{richmen}(#1,#2)}

%% /macros %%

\title{Rule-based specification of the blockchain logic}
\author{Erik de Castro Lopo, Nicholas Clarke \& Arnaud Spiwack}

\begin{document}

\maketitle

\tableofcontents

\section*{Status}
\label{sec:status}

\begin{description}
\item[Draft 0, Mar 14, 2018 (Arnaud)] Outline
\item[Draft 1, Mar 20, 2018 (Arnaud)] First version of the block
  validation rules
\item[Draft 2, Mar 28, 2018 (Arnaud)] Slot leader selection rules
\item[Draft 3, Apr 9, 2018 (Nicholas)] Review by Nicholas Clarke
\item[Draft 4, Apr 13, 2018 (Arnaud)] Partial specification of stake
  delegation and longest chain selection
\item[Draft 5, Apr 17, 2018 (Arnaud)] Specification of delegation
\item[Draft 6, May 16, 2018 (Arnaud)] Specification of mempool,
  description of parameter, various accuracy improvement, appendix
\item[Draft 7, May 31st, 2018 (Arnaud)] Specification of block
  issuance, specification of secret sharing
\item[Draft 8, Jun 01, 2018 (Nicholas)] Editing
\end{description}

\section{Introduction}
\label{sec:introduction}

This document describes the rules underpinning the blockchain data
structure of Cardano's settlement layer. These rules are described
independently of any network or communication mechanism: it is assumed
that some mechanism provides us with, \emph{e.g.}, blocks, and that
our task is to assume that they are accepted or not. Notably, we make
no distinction between rejecting a block and storing it for later use.

A direct consequence of ignoring the communication mechanisms, is that
we do not have to serialise data, hence data is represented in
abstract syntax, with no reference to a binary representation.

As a further simplification, we ignore any unknown data, unparsed
parameters, and versioning issues. This document specifies the system
as it is.\improvement{We may want to add something about protocol
  updates, though.}

\section{Terminology \& notations}
\label{sec:notations}

\begin{description}
\item[Lists] Given a set $A$, we write $\lists{A}$ for the set of lists of
  elements of $A$, and $\nelists{A}$ for the set of non-empty lists of elements
  of $A$. For an element $a∈A$ we write simply $a$ for the singleton list. For
  two lists $l,r∈\lists{A}$, we write $l,r$ for their concatenation. We write
  $\emptylist$ for the empty list. For a list $l∈\lists{A}$, we write
  $\length{l}$ for its length. We write $\pos{i}{l}$ for the $i$-th element of
  the list (the head of the list is $\pos{1}{l}$). We sometimes abuse notation
  and write $x∈l$ for $l = l',x,l''$.

\item[Support] Let $f$ be a function $f∈X⟶R$ where $R$ is a
  commutative monoid (typically the additive monoid of $ℕ$, $ℤ$, $ℚ$,
  $ℝ$, …), the \emph{support} of $f$ is $\{ x∈X | f(x)≠0 \}$. We say
  that $f$ has finite support, or that $f$ is a function with finite
  support, if the support of $f$ is a finite set. If $f$ is a finite
  support function, we write $\supportsize{f} ∈ ℕ$ for the cardinal of
  the support of $f$.

  A notable property of finite-support function, is that they can be
  summed: $∑_{x∈X}f(x)$ is well-defined. Indeed, it suffices to sum
  over the the (finite!) support of $f$, which, as a finite sum, is
  defined.

  Additionally, for $f ∈ X ⟶ R$ a finitely supported function and
  $g ∈ X -> Y$ an arbitrary function, there is a function
  $\extension{g}{f} ∈ Y ⟶ R$, called the extension of $f$ along $g$,
  defined as
  $$
  \extension{g}{f}(y) = ∑_{g(x)=y} f(x)
  $$
  Note that $\extension{g}{f}$ is a total function even is $g$ is partial
  (see the definition of partial functions below).

  \begin{implementationhint}
    Finite support function are typically implemented as finite maps
    (as in Haskell's \verb+Data.Map+ module), where we then interpret
    absent bindings as mapping to $0$ (\emph{e.g.} using
    \verb+Map.findWithDefault 0+). There are subtleties if we want to
    test equality as natively, such data structure distinguishes
    bindings to $0$ and absence of bindings. One way to solve this is
    to remove any binding to $0$ from the map.
  \end{implementationhint}

\item[Partial functions] A partial function $f$ from $A$ to $B$ is a
  pair of a subset $U∈\powerset{A}$, called the \emph{domain} of $f$,
  and a function $U⟶B$. The set of partial functions from $A$ to $B$
  is written $A ⇀ B$. We write $f(a)$ for the image of $a∈U$.

  We write $f⊆g$, for partial functions $f$ and $g$ with respective domains $U$
  and $V$, if $U ⊆ V$ and $∀a∈U. f(a)=g(a)$. When $f⊆g$, we write
  $g⧵f$ for the partial function with support $V⧵U $ defined as:
  $$
  g⧵f(a) = g(a)
  $$

  For two partial functions $f$ and $g$ with respective domains $U$
  and $V$, such that $∀a∈U∩V. f(a)=g(a)$, we write $f⊎g$ for the partial function
  of domain $U∪V$ defined as:
  $$
  f⊎g(a) = \left\{
    \begin{array}{ll}
      f(a) & \mbox{if }a∈U\\
      g(a) & \mbox{if }a∈V
    \end{array}
    \right.
  $$
  Note that if $U∩V=∅$, the condition $∀a∈U∩V. f(a)=g(a)$ holds
  vacuously, hence $f⊎g$ is defined.\unsure{Maybe a better symbol for
    this ``gluing'' union.}

  We say that a partial function has finite domain if its domain is
  a finite subset.

  \begin{implementationhint}
    A partial function $A⇀B$ can be implemented as a function
    \verb+A -> Maybe B+, provided that its domain is decidable. Mostly we are
    interested in finite domain partial function which are faithfully
    represented by finite maps (in the sense of \verb+Data.Map+). The $f⊎g$
    gluing union can be implemented using \verb+Data.Map.Merge.mergeA+.
  \end{implementationhint}


\item[Remainder] For any $a∈ℤ$ and $b∈ℕ$, let us write $\rem{a}{b}$
  for the unique $0⩽r<b$ such that $a=qb+r$.

\item[Projection] For a cartesian product $X=X_1×…× X_n$ with element $x =
  (x_1, \dots, x_n)$, we define the \textit{projection onto the $i^{th}$ component}
  $\pi_i∈X⟶X_i$ as \[\pi_i(x)=x_i,\] for all $i \leq n$.
\end{description}

\section{Definitions}
\label{sec:definitions}

\begin{description}
\item[System of identifiers] A set $A$ is said to \emph{have
    identifiers} if there is a set, written $\idsof{A}$ and an
  injective function $\id ∈ A ⟶ \idsof{A}$.

  When an item $a∈A$ such that $\id(a) = h$ is clear from the context,
  we will often identify $a$ and $h$.

  \begin{implementationhint}
    In implementations, identifiers are typically realised by
    cryptographic hash functions, which are injective for all intents
    and purposes.
  \end{implementationhint}

\item[Agent id] We assume a set $\agentids$ of identifiers
  representing the agents interacting on the blockchain. Furthermore
  $\agentids$ is assumed to come with a total order, which we write as
  $⩽$.
  \begin{implementationhint}
    In a concrete implementation agent ids would be public
    cryptographic keys and signatures, depending on how what the id is
    used for. In an executable specification, agent ids can be
    anything which uniquely identifies an agent; let's call it an
    \emph{abstract} agent id. When translating from an executable
    specification to concrete blocks, we need to maintain a mapping of
    abstract agent ids to public-private key pairs, so that we can
    sign what needs to be signed.
  \end{implementationhint}
  
  \begin{currentvalue}
    In Cardano, the total order of $\agentids$ is the lexicographic
    ordering on the public keys.
  \end{currentvalue}\unsure{We are actually not completely
      sure about this. This is tracked by
      \href{https://iohk.myjetbrains.com/youtrack/issue/CDEC-163}{CDEC-163}.}

\item[Stake distribution] A \emph{stake distribution} is a function
  $d ∈ \agentids ⟶ ℕ$ with finite support. We write
  $\stakedistributions$ for the set of stake distributions.

\item[Slot] An integer.

\item[Seeds] We assume a commutative monoid $\seeds$ of \emph{seeds}. The monoid
  operations are written multiplicatively.

\item[Stake repartition] A stake repartition is a function
  $r ∈ \stakerepartitions ⊆ \agentids ⟶ \mathbb{Q}$ with finite
  support, and such that $r(a) ⩾ 0$ for all $a$. And such that
  $∑_{a∈\agentids} r(a) = 1$.

  Repartition are used as part of transaction output to allocate stake
  to parties which are not necessarily the recipient of the
  output. Agent $a$ will receive stake in proportion $r(a)$ (with
  subtleties due to rounding, see
  Section~\ref{sec:derived-functions}).

\item[Transaction] Transactions have a system of identifiers
  $\txids$.

  A transaction $t∈\txs$ consists of
  \begin{record}
  \item[Inputs] A non-empty list of pairs in
    $\nelists{(\txids × ℕ)}$.

    An input $in = (tx, ix) ∈ \txids × ℕ$ should be read as a pointer to an (expectedly unspent)
    output, addressed by the transaction $tx$ in which it was created, and
    its index $ix$ in the list of outputs of $tx$. We write
    \begin{itemize}
    \item $\intrans{in} = tx$
    \item $\inindex{in} = ix$
    \end{itemize}

  \item[Outputs] A non-empty list of triples in
    $\nelists{(\agentids × ℕ × \stakerepartitions)}$.\improvement{This
      only embodies pay-to-public-key schemes. For pay-to-script and
      such, we will need further refinements.}

    For an output $out = (id, q, r)$, we write
    \begin{itemize}
    \item $\recipient{out} = id$
    \item $\amount{out} = q$
    \item $\repartition{out} = r$
    \end{itemize}

  \item[Witnesses] A non-empty list $w∈\nelists{\agentids}$ of agent
    ids. In an implementation these ids would be realised by
    cryptographic signatures.
  \end{record}

\item[Delegation graph] A delegation graph is a function $g ∈
  \agentids ⟶ \agentids$. The initial delegation graph is defined as
  the identitity function $\initialdelegation(a) = a$.

  Given a delegation graph $g$, we write $\transitivefix{g}$ for the
  partial function defined as
  $$
  \transitivefix{g}(a)=a' \iff ∃n. g^n(a) = a'∧ g(a') = a'
  $$

\item[Block] Blocks have a system of identifier $\blockids$. There is
  a distinguished $b_0∈ \blockids$, called the genesis identifier,
  which is not in the image of $\id∈ \blocks ⟶ \blockids$.

  Block come in two flavours \emph{regular blocks} and \emph{epoch
    boundary blocks}, defined below.

\item[Regular block]
  A regular block $b∈\rblocks ⊆ \blocks$
  consists of
  \begin{record}
  \item[predecessor] A block identifier
  \item[slot number] The slot at which the block has been issued
  \item[issuer] The identifier of the agent who issued this block. In
    an implementation, this would be realised by a cryptographic
    signature.
  \item[transactions] A list of transactions
  \item[delegation certificates] A list of pairs in
    $\lists{(\agentids×\agentids)}$. For a delegation certificate
    $c = (D,d)∈\agentids×\agentids$, we write
  \begin{itemize}
  \item $\delegator{c}=D$
  \item $\delegatee{c}=d$
  \end{itemize}

  \begin{implementationhint}
    In a concrete implementation, these certificates would be
    cryptographically signed by $D$, to ensure that $D$ is indeed
    authorising $d$ to sign blocks for them.
  \end{implementationhint}

  \item[seed computation] Any of the following
    \begin{description}
    \item[Nothing] An element of the singleton type $\{⋆\}$.
    \item[Commitment] A finite-domain partial function $\agentids⇀\seeds×\seeds$.
      \begin{implementationhint}
        The idea here is that the first seed is the commitment, in the
        form of a partial function from agent id to encrypted shares
        (in the sense of secret sharing), while the second message is
        a proof that the shares are correct.

        In this specification, we intend the two messages to be equal,
        but this would be replaced, in the implementation, by the
        statement that the proof validates the shares.
      \end{implementationhint}
    \item[Opening] A finite-domain partial function $\agentids⇀\seeds$.
    \item[Recovery] A finitely supported function
      $r∈\agentids→\agentids⇀\seeds$, such that for all $a$ in the
      domain, $r(a)$ has finite domain (partial functions are a commutative monoid
      under $⊎$. This monoid is actually partial, but we will ensure
      that it is only used when defined).

      \begin{implementationhint}
        We want to understand $r(a)(a')$ as the decryption (by agent $a$) of the
        share created by agent $a'$ and sent to agent $a$. So $r(a)$ is the set
        of all the shares that $a$ decrypted.

        In the specification, we, again, implement the decrypted
        shares as the whole, plain text, seed.
      \end{implementationhint}
    \end{description}

  \end{record}
\item[Epoch boundary blocks] An epoch boundary block
  $b∈\bblocks⊆\blocks$ consists in
  \begin{record}
  \item[predecessor] A block identifier
  \item[slot number] The slot at which the boundary takes place
  \end{record}

  \begin{implementationhint}
    In Cardano, boundary blocks contain a lot more information but (up
    to the injectivity-of-hashes hypothesis) this information is
    determined by the above. It will affect however the actual hash of
    the epoch boundary block, it is therefore infeasible to concretise
    an epoch boundary block without knowing the blockchain up to the
    predecessor. The concretisation of epoch boundary block is
    therefore not a pure function.
  \end{implementationhint}

\end{description}

\section{Parameters}
\label{sec:parameters}

\subsection{Validation parameters}
\label{sec:validation-parameters}

This section lists parameters of the validation rules. The concrete
instantiation of most of these parameters in Cardano is specified in
Section~\ref{sec:params}. They are kept abstract in the validation
rules to highlight that the rules can be stated independently of their
concrete value, which simplifies the specification.

\begin{description}
\item[Slot leader] We assume a function
  $\slotleader{⋅}{⋅}{⋅} ∈ \stakedistributions×\seeds×ℕ ⟶ \agentids$,
  which, given a stake distribution and a source of randomness, maps
  each subsequent slot to an agent id. This function is defined in
  Section~\ref{sec:slot-leader}.

\item[Current slot] The current slot is a parameter of the validation
  rules desribed in this document (\emph{i.e.} it doesn't make sense
  in general to ask whether, say, a chain is valid, only that it is
  valid \emph{at the current slot}).

  This parameter is not instantiated in Section~\ref{sec:params}
  because it actually varies. In an implementation, the current slot
  would be a function of the wall-clock time. The role of the current
  slot is to prevent attacks where a malicious slot leader at slot
  $i+1$ would try to issue a block without acknowledging the block
  issued by the slot leader at slot $i$.

\item[Height] We assume a height function
  $\height{⋅} ∈ \lists{\blocks} ⟶ ℕ$. It is usually necessary, for the
  sake of good behaviour of the system to assume that
  $\height{b₁,…,b_n} < \height{b₁,…,b_n,b_{n+1}}$, though it is not
  necessary for the specification itself. A typical such function is
  the length of the list, though more refined criterion are usually
  chosen. This function is defined in Section~\ref{sec:height}.

\item[Chain quality] We assume a function $\quality{⋅} ∈ ℕ ⟶ ℕ$, which
  maps a number of blocks into a number of blocks. The meaning of the
  chain qualitity function is that, assuming that a block at depth $k$
  will never be rolled back (see Section~\ref{sec:conf-param}), then
  the chain will never roll back more than $\quality{k}$ slots. This
  function is defined in Section~\ref{sec:chain-quality}.\unsure{I put
  the chain quality here because it is at a rather prominent place in
  the code, it is never directly used in the specification, rather the
  phases below are. So we may skip this and only introduce chain
  quality in Section~\ref{sec:params}.}

\item[Secret-sharing phases] We assume three subsets of $ℕ$
  $$
  \commitphase, \openphase, \recoveryphase ∈ \powerset{ℕ}
  $$
  These are intended as subsets of the slots within an epoch. They are
  expected to be intervals, in that for all $a<b<c$ such that
  $a∈\commitphase$ and $c∈\commitphase$, we have $b∈\commitphase$
  (respectively $\openphase$, $\recoveryphase$). It is also expected
  that for $x ∈ \commitphase$, $y∈\openphase$, and $z∈\recoveryphase$,
  we have $x<y<z$. These subsets are defined in
  Section~\ref{sec:vss-phases}.

  \improvement{This paragraph is pretty horrible to read, just because the
    subscripts are so long.}
\end{description}

\subsection{Configuration parameters}
\label{sec:conf-param}

This section document parameters of nodes, which are set via
configuration arguments.

\begin{description}
\item[Maximum mempool size] We assume a number $\maxmempoolsize ∈ ℕ$(
  representing the maximum allowed size for a mempool (see
  Section~\ref{sec:entering-mempool}).
  \begin{currentvalue}
    At time of writing, in the implementation of Cardano, the maximum
    mempool size is 200.
  \end{currentvalue}
\end{description}

\subsection{Chain parameters}
\label{sec:chain-parameters}

This section lists the parameters of a Cardano block chain. They are
specified in the initial state and can be further modified by updates.

\begin{description}
\item[k] This entire specification is parameterised by a natural
  number $k$. It comes from the Ouroboros paper: it is the depth after
  which a block is considered stable, and will never rollback under
  any circumstance.

  \begin{currentvalue}
    In Cardano, $k=2160$
  \end{currentvalue}

\item[Initial stake distribution] We assume an \emph{initial stake
    distribution} $\initialdistribution ∈ \stakedistributions$.

  Addresses in the support of the initial stake distribution are
  referred to as \emph{redeem address}. These address can't receive
  new coins in the chain.\info{This is not entirely accurate. In the
    Cardano data types there are Redeem address which are not in the
    support of the initial distribution, nothing can be done with
    them. It may be more accurate to split the type of agent ids in
    two disjoint subsets.}

\item[Initial seed] A distinguished seed $\initialseed ∈ \seeds$.

\item[Minimum fee] We assume a function $\minimumfee{⋅} ∈ \txs ⟶ ℕ$,
  called the minimum fee function.\info{It's not clear that the
    minimum fee, as is currently implemented, is indeed a function of
    the (abstract represention of) the transaction. This may create
    mismatches in tests.}

\item[Initial transactions] We assume a finite subset $\txinit ⊆ \txs$
  of transactions. The intent behind $\txinit$ is that these
  transactions are present ``in the initial state''. They represent
  the initial assets in the system. Their inputs are
  meaningless.\improvement{We may want to compute the initial
    distribution based on $\txinit$.}

\item[Epoch] We assume a pair or functions $\epoch{⋅}∈ℕ⟶ℕ$ and
  $\epochslot{⋅}∈ℕ⟶ℕ$, the \emph{epoch decomposition} of a slot, such
  that~--~writing $(\epoch{⋅}, \epochslot{⋅})$ for the function, in
  $ℕ⟶ℕ×ℕ$, which maps a slot $s$ to
  $(\epoch{s}, \epochslot{s})$~--~$(\epoch{⋅}, \epochslot{⋅})$ is a
  strictly increasing function, from the natural ordering of $ℕ$ to
  the lexicographic product of the natural ordering of $ℕ$ with
  itself.\improvement{We may want to additionally impose that the
    inverse image of $e×ℕ$ is downward closed.}\improvement{Layout is
    broken in this paragraph.}\improvement{It will probably be clearer
    to give a name to $(\epoch{⋅}, \epochslot{⋅})$ rather than
    referring to it symbolically. And define the two projection in
    terms of this function, like is done in the rest of the text,
    rather than the other way around.}

  \begin{currentvalue}
    For reference, $\epoch{s}$ and $\epochslot{s}$, are defined, in
    the bootstrap era of Cardano, of as the unique numbers such that
    $s=\epoch{s}\cdot 10k+\epochslot{s}$ and $0⩽ \epochslot{s} < 10k$.
  \end{currentvalue}


\item[Bootstrap stakeholders] We assume a finite set
  $\bootstrapstakeholders ⊆ \agentids$. The bootstrap stakeholders receive all the
  stake during the bootstrap era.\unsure{For the sake of simplicity, I
    assume, for the time being, that the bootstrap stakeholders all
    get the same share of stake of each output. This is not necessary
    in cardano, but it is the case in practice. We can refine later if
    we so wish}

  We assume a stake repartition
  $\bootstraprepartition ∈ \stakerepartitions$ which allocates stake
  to each bootstrap stakeholder in equal proportion. That is
  $$
  \bootstraprepartition(a) = \left\{
    \begin{array}{cl}
      \displaystyle{\frac{1}{\supportsize{r}}}& \mbox{if $a∈\bootstrapstakeholders$}\\
      0 & \mbox{otherwise}
    \end{array}\right.
  $$

  We will also assume a function $\selectrichman{⋅} ∈ ℕ ⟶ \bootstrapstakeholders$
  which selects one of the bootstrap stakeholder based on the value of an unspent
  output. In a concrete implementation this function needs to have
  good randomness properties.

\end{description}
\section{Extending a chain}
\label{sec:adding-block}

This section specifies whether a block can extend a given chain. That
is we have one version of history (forks are handled in
Section~\ref{sec:chain-selection}), and we are receiving a block. If
the block passes the validation rules of this section, then the
history is one block longer.

A sequence $b₁,…,b_n$ is said to be a \emph{valid chain} if
$b₁,…,b_{n-1}$ is a valid chain,
and $b_n$ is a \emph{valid extension} of $b₁,…,b_{n-1}$.

\subsection{Boundary block}
\label{sec:boundary-block}

A boundary block $b∈\bblocks$ is a valid extension of a valid chain
$b_1,…,b_n$ if all of the following hold
\begin{description}
\item[Predecessor] The predecessor of $b$ is $b_n$ or, if $n=0$, the predecessor
  of $b$ is the genesis identifier $b₀$.
\item[Not future slot] The slot number is no larger than the current
  slot.
\item[Not past slot] The slot number of $b$ is larger than the slot.
  number of $b_n$
\item[At boundary] Let $s$ be the slot number of $b$, $\epochslot{s} =
  0$ (note that it implies, in particular, that $\epoch{b_n}<\epoch{s}$).
\end{description}

\subsection{Regular block}
\label{sec:regular-blocks}

A regular block $b∈\rblocks$ is a valid extension of a valid chain
$b_1,…,b_n$ if all of the following hold
\begin{description}
\item[Predecessor] The predecessor of $b$ is $b_n$ or, if $n=0$, the predecessor
  of $b$ is the genesis identifier $b₀$.
\item[Not future slot] The slot number is no larger than the current
  slot.
\item[Not past slot] The slot number of $b$ is larger than the slot.
  number of $b_n$
\item[No boundary-crossing] Let $s_b$ be the slot number of $b$ and
  $s_n$ be the slot number of $b_n$, then $\epoch{s_b}=\epoch{s_n}$
  (this condition imposes that epochs be started by epoch boundary blocks).
\item[Issuer] Let $i$ be the issuer of $b$, and let $l$ be the slot
  leader at the slot of $b$:
  $$
  l = \slotleader{\epochdistribution{b_1,…,b_n}}{\epochseed{b_1,…,b_n}}{\epochslot{slot}}
  $$
  See Section~\ref{sec:derived-functions} for the definitions of
  $\epochdistribution{⋅}$ and $\epochseed{⋅}$.
  \improvement{this doesn't read too well, improve.}

  Then, $i$ must be transitively authorised by $l$ in that:
  $$
  i = \transitivefix{\delegations{b₁,…,b_n}}(l)
  $$
  Note that, by definition, there is (at most) a single agent
  authorised by $l$ (it may be $l$ itself, if no delegation certificate
  has been issued). See Section~\ref{sec:derived-functions} for the
  definition of $\delegations{⋅}$.
\item[Transactions] Each transaction in the block's transactions is
  validated according to the following rules.
  \begin{description}
  \item[Stateless]\hfill
    \begin{description}
    \item[Valid outputs] All the outputs of the transaction must have
      amount $a > 0$.
    \item[No double-spending] The list of inputs contains no duplicate.
    \item[No redeem output] For every $out∈\outputs{τ}$,
      $\initialdistribution(\recipient{out})=0$.
    \end{description}
  \item[Stateful] Stateful transaction validation rules for a transaction $τ$
    are parametrised by a list of transactions $τ_1,…,τ_m$ which we will abridge
    as $\overline{τ}$.

    Given a valid chain $b₁,…,b_n$ and $b$, the block being validated, such
    that $\transactions{b}=t₁,…,t_k$. Then, for the purpose of
    stateful validation of the transaction $t_i$, we set
    $\overline{τ} = \transactions{b₁},…,\transactions{b_n},t₁,…,t_{i-1}$.
    \begin{description}
    \item[Unspent inputs] Each input of the transaction $τ$ is
      unspent. That is for each $i ∈ \inputs{τ}$
      \begin{itemize}
      \item $\intrans{i}∈\overline{τ}$ or $\intrans{i} ∈ \txinit$
      \item $\inindex{i} ⩽ \length{\outputs{\intrans{i}}}$
      \item for each $t ∈ \overline{τ}$, $i ∉ \inputs{t}$.
      \end{itemize}

      In these conditions, we can extend the function $\recipient{⋅}$
      and $\amount{⋅}$, as well as $\repartition{⋅}$, on the $i$ as
      \begin{itemize}
      \item $\recipient{i} = \recipient{\pos{\inindex{i}}{\outputs{\intrans{i}}}}$
      \item $\amount{i} = \amount{\pos{\inindex{i}}{\outputs{\intrans{i}}}}$
      \item $\repartition{i} = \repartition{\pos{\inindex{i}}{\outputs{\intrans{i}}}}$
      \end{itemize}
      \info{Remark: in all this rule, we use implicitly
        $\overline{τ}$, to cast $\intrans{i}∈\txids$ to an actual
        transaction. }

    \item[Conservation of value] The total amount of outputs must not exceed
      the total amount of inputs. A minimum fee must be paid.
      $$
      \minimumfee{τ} + ∑_{o∈\outputs{τ}} \amount{o}⩽ ∑_{i∈\inputs{τ}} \amount{i}
      $$

      If for every $i∈\inputs{τ}$,
      $\initialdistribution{\recipient{i}}>0$, that is all the inputs
      are from redeem addresses, then no fee need be paid, and the
      rules become:
      $$
      ∑_{o∈\outputs{τ}} \amount{o}⩽ ∑_{i∈\inputs{τ}} \amount{i}
      $$
    \item[Authorised] The list of witnesses has the same length as the
      list of inputs
      $$
      \length{\witnesses{τ}} = \length{\inputs{τ}}
      $$
      and for each input, the corresponding witness is the agent id of
      the input
      $$
      ∀ 0<k⩽\length{\inputs{τ}}. \recipient{\pos{k}{\inputs{τ}}} =
      \pos{k}{\witnesses{τ}}
      $$
    \end{description}

  \item[Bootstrap era] The following conditions must all pass to
    validate transactions in the bootstrap era. This specification only
    describes the behaviour during the bootstrap era, hence the
    following conditions are checked unconditionally.
    \begin{description}
    \item[Bootstrap stake delegation] For every output $o∈\outputs{τ}$, the
      stake is delegated in to the bootstrap stakeholders:
      $\repartition{o} = \bootstraprepartition$.

      \begin{treacherous}
        This is slightly imprecise because the bootstrap repartition
        has two representations in Cardano's blockchain: it has a
        name (a data constructor), but it can also be represented by a
        explicit map. Only the former is actually permitted
      \end{treacherous}

    \end{description}
  \end{description}

\item[Delegation certificates] Let
  $\overline{c}=\delegationcertificates{b}$ be the list of delegation
  certificates of $b$. Then for any two $c₁,c₂∈\overline{c}$, it must
  hold that $\delegator{c₁}≠\delegator{c₂}$.

  Note that this prevents any agent which has issued a delegation certificate to
  another party from issuing any block: they \emph{must} be signed by the
  delegatee (or, transitively, by the agent authorised to sign instead of the
  delegatee). That is, until the agent issues a new delegation certificate to
  itself, which, in effect, revokes any previous delegation certificate. See
  also the definition of $\transitivefix{g}$ in Section~\ref{sec:definitions}.

\item[Seed computation] This rule divides into different cases, depending on the
  content of the seed computation field of the block $b$ (the functions
  $\richmen{⋅,⋅}$, $\epochcommitments{⋅}$, $\epochopens{⋅}$, and
  $\epochrecoveries{⋅}$ are defined in Section~\ref{sec:derived-functions}).
  \begin{description}
  \item[Nothing] If there is no message, then no further rule must be
    enforced.
  \item[Commitment] If there is a commitment message $c ∈
    \agentids⇀\seeds×\seeds$ then, and calling its domain $U$,
    \begin{description}
    \item[Commitment phase] Let $s$ be the current slot, then we must
      have $\epochslot{s}∈\commitphase$.
    \item[Richmen] We must have
      $U⊆\richmen{\delegations{b_1,…,b_n}}{\currentdistribution{b_1,…,b_n}}$.
    \item[Valid] For every $a∈U$, let $(m,p) = c(a)$, we must have
      $m=p$.
    \item[Fresh] Let $V$ be the domain of $\epochcommitments{b_1,…,b_n}$,
      we must have $U∩V=∅$.
    \end{description}
  \item[Open] If there is an opening message $o ∈
    \agentids⇀\seeds$ then, calling its domain $U$,
    \begin{description}
    \item[Opening phase] Let $s$ be the current slot, then we must
      have $\epochslot{s}∈\openphase$.
    \item[Valid] We must have that $o ⊆ π_1 ∘
      \epochcommitments{b_1,…,b_n}$.
    \item[Fresh] Let $V$ be the domain of $\epochopens{b_1,…,b_n}$,
      we must have $U∩V=∅$.
    \end{description}
  \item[Recovery] If there is a recovery message $r ∈
    \agentids⟶\agentids⇀\seeds$ then, calling its support $U$,
    \begin{description}
    \item[Recovery phase] Let $s$ be the current slot, then we must
      have $\epochslot{s}∈\recoveryphase$.
    \item[Valid] For every $a∈U$, we must have that $r(a) = (π_1 ∘ \epochcommitments{b_1,…,b_n})⧵\epochopens{b_1,…,b_n}$.
    \item[Fresh] Let $V$ be the domain of $\epochrecoveries{b_1,…,b_n}$,
      we must have $U∩V=∅$.
    \end{description}

  \end{description}


\end{description}

\section{Derived functions}
\label{sec:derived-functions}

This section defines functions which are only defined on valid chains,
per the definition of Section~\ref{sec:adding-block}. The functions
defined in this sections are also used in
Section~\ref{sec:adding-block}. Therefore this section and
Section~\ref{sec:adding-block} are mutually recursive.

Functions which project fields from regular blocks are extended to
epoch boundary blocks by returning the natural trivial value. For
instance, for an epoch boundary block $b$,
$\transactions{b}=\emptylist$

\begin{description}
\item[Last epoch index] For a valid chain, we write
  $$
  \epochindex{b_1,…,b_n} = \max_{\mathclap{\substack{
      0<i<n\\
      \epoch{b_i} < \epoch{b_n}
  }}} i
  $$
  for the index in the chain (not the slot) of the last block of the
  previous epoch (the current epoch is taken to be the epoch of $b_n$,
  the tip of the chain).

  If there is no complete epoch (in particular if the chain is empty)
  we set $\epochindex{b_1,…,b_n} = 0$.
\item[Stake distribution] Let us define
  $\currentdistribution{τ₁,…,τ_n}$ by induction, for a valid sequence
  of transactions\improvement{Valid sequence of transaction hasn't
    properly been defined, but it's implicit in the definition above,
    it would maybe be worth making more precise.}:
  \begin{itemize}
  \item $\currentdistribution{\emptylist} = \initialdistribution$
  \item Inductively, and writing
    \begin{itemize}
    \item $⌊x⌋$ for the integral part (aka floor) of $x$; \emph{i.e.}
      $⌊x⌋$ is the unique integer such that $⌊x⌋ ⩽ x < ⌊x⌋+1$
    \item $\kronecker{⋅}{⋅}$ for
      \href{https://en.wikipedia.org/wiki/Kronecker_delta}{Kronecker's
        delta} function:
      $$
      \kronecker{i}{j} = \left\{
        \begin{array}{ll}
          1 & \mbox{if $i=j$}\\
          0 & \mbox{otherwise}
        \end{array}
      \right.
      $$
      In particular, $\kronecker{i}{j}q = q$ if $i=j$ and $0$
      otherwise. Kronecker's delta is used to select summands: it is
      the arithmetician's \verb+ifThenElse+ combinator.
    \item $\stakeof{o}{a} = \repartition{o}(a)\cdot\amount{o} ∈ \mathbb{Q}$
    \item For inputs of $τ_{n+1}$, we extend $\stakeof{⋅}{⋅}$ to
      inputs using the validity hypothesis to define $\repartition{i}$ and $\amount{i}$
    \end{itemize}
    We define
\newcommand{\stakerepartitionformula}[4]{∑_{#2∈#3} \left(
            \begin{aligned}
              &⌊\stakeof{#2}{#1}⌋\\
              &\qquad+\\
              &\raisebox{0pt}[0pt][3.5ex]{$\displaystyle{\kronecker{#1}{\selectrichman{\amount{#2}}}\left(\amount{#2} - ∑_{#4∈\agentids}
            ⌊\stakeof{#2}{#4}⌋\right)}$}
            \end{aligned}\right)}
    \begin{align*}
      \makebox[1em][l]{$\currentdistribution{τ₁,…,τ_n, τ_{n+1}}(a) =$}\\
      & \phantom{=} \currentdistribution{τ₁,…τ_n}(a)\\
      & - \stakerepartitionformula{a}{i}{\inputs{τ_{n+1}}}{b}\\
      & + \stakerepartitionformula{a}{o}{\outputs{τ_{n+1}}}{b}
    \end{align*}

    That is, the contribution of unspent output $o$ to agent $a$ is
    the proportion of the value of $o$ specified by the stake
    repartition of $o$ rounded \emph{down} to the nearest integer. The
    remaining unallocated stake goes to the bootstrap stakeholder
    specified by the $\selectrichman{⋅}$ function, applied to the
    total value of $o$.

    \begin{treacherous}
      The treatment of remainder as specified here is only valid for
      the bootstrap era. Outside of the bootstrap era, the remainder
      may be different, hence we may have to calculate remainder
      differently for transactions of the bootstrap eras, and
      transaction posterior to the bootstrap eras. See
      Appendix~\ref{sec:delegation-shelley}.
    \end{treacherous}
  \end{itemize}

  We can then extend $\currentdistribution{⋅}$ on valid chains by defining
  $$
  \currentdistribution{b₁,…,b_n} = \currentdistribution{\transactions{b₁},…,\transactions{b_n}}
  $$
  Finally we define
  $$
  \epochdistribution{b₁,…,b_n} = \currentdistribution{b₁,…,b_{\epochindex{b_1,…,b_n}}}
  $$

\item[Richmen] For a delegation graph $g$ and a stake distribution
  $d∈\stakedistributions$, note that $\extension{g}{d}$ is the stake
  distribution corresponding to $d$ except that all the stake of an
  agent $a$ has been (transitively) passed to its delegatee.

\newcommand{\threshold}[1]{\mathsf{threshold}(#1)}
  We define
  $$
  \richmen{g}{d} = \{ a ∈ \agentids~|~\extension{g}{d}(a) ⩾
  \threshold{∑_{b∈\agentids} d(b)} \}
  $$
  \improvement{TODO: declare the threshold function as a parameter.}

\item[Seed computation]\hfill
  \begin{itemize}
  \item Let $c_1,…,c_k$ be the list of partial functions
    $\agentids⇀\seeds×\seeds$ extracted from the seed computation
    field of the blocks in $b_{\epochindex{b_1,…,b_n}+1},…,b_n$ which
    are commitment messages, then define
    $$
    \epochcommitments{b_1,…,b_n} = c_1 ⊎ … ⊎ c_k
    $$
  \item Let $o_1,…,o_k$ be the list of partial functions
    $\agentids⇀\seeds$ extracted from the seed computation
    field of the blocks in $b_{\epochindex{b_1,…,b_n}+1},…,b_n$ which
    are opening messages, then define
    $$
    \epochopens{b_1,…,b_n} = o_1 ⊎ … ⊎ o_k
    $$
  \item Let $r_1,…,r_k$ be the list of partial functions
    $\agentids⇀\agentids⇀\seeds$ extracted from the seed computation
    field of the blocks in $b_{\epochindex{b_1,…,b_n}+1},…,b_n$ which
    are recovery messages, then define
    $$
    \epochrecoveries{b_1,…,b_n} = r_1 ⊎ … ⊎ r_k
    $$
  \end{itemize}
\item[Seed] Let $U$ be the domain of the partial function
  $$
  r = ⨄_{a∈\agentids}\epochrecoveries{b_1,…,b_{\epochindex{b_1,…,b_n}}}(a)
  $$
  and $V$ be the domain of the partial function $\epochopens{b_1,…,b_{\epochindex{b_1,…,b_n}}}$.

  Then define
  $$
  \left\{
  \begin{array}{lcll}
    \epochseed{b_1,…,b_n}&=& \initialseed & \mbox{if $\epochindex{b_1,…,b_n}=0$}\\
    \epochseed{b_1,…,b_n}&=&∏_{a∈V}\epochopens{b_1,…,b_{\epochindex{b_1,…,b_n}}}(a) & \mbox{otherwise}\\
                         &×&∏_{a∈U}r(a)
  \end{array}
  \right.
  $$
\item[Delegation graph] Let us define $\delegations{c₁,…,c_n}$, for a
  list of delegation certificates, by induction:
  \begin{itemize}
  \item $\delegations{\emptylist} = \initialdelegation$
  \item $$
    \delegations{c₁,…,c_n,(D,d)}(a) = \left\{
      \begin{array}{ll}
        d & \mbox{if $a=D$}\\
        \delegations{c₁,…,c_n}(a) & \mbox{otherwise}
      \end{array}\right.
      $$
  \end{itemize}

  We can extend $\delegations{⋅}$ on valid chain by defining
  $$
  \delegations{b₁,…,b_n} = \delegations{\delegationcertificates{b₁},…,\delegationcertificates{b₁}}
  $$

\end{description}

\section{Instantiating parameters}
\label{sec:params}

\subsection{Height}
\label{sec:height}

The height of a chain is defined as:

$$
\height{b₁,…,b_n} = n
$$

\subsection{Slot leader}
\label{sec:slot-leader}

A \emph{number generator} is a function $f ∈ ℕ ⟶ S ⟶ ℕ×S$ for some set
$S$, such that for every $n$ and $s$, $(i,s') = f(n)(s) ⟹ 0 ⩽ i <
n$. For security, we would need to assume that $f(n)$ has good
randomness properties, but we can safely ignore this aspect for the
purpose of this document.

In the rest of the section, we assume given a number generator $f$, a
seed $s_0∈S$, and a stake distribution $d$. Let $N$, the \emph{total
  stake}, be $N=∑_{a∈\agentids} d(a)$ (which is defined since $d$ has
finite support).

Let us define the sequences $(s_n)_{n∈ℕ}$ and $(u_n)_{n>0}$ as
\begin{itemize}
\item $s_0$ is as given above
\item $(u_{n+1}, s_{n+1}) = f(N)(s_n)$
\end{itemize}

We can then define the slot leader function as

$$
\slotleader{d}{s_0}{n} = \min_{l∈\agentids} \left( u_n < ∑_{a⩽l} d(a) \right)
$$

That is, laying the agents in order on a line, and letting each take
$d(a)$ unit of length. We select the agent whose extent on the line
contains $u_n$.

\subsection{Number generator}
\label{sec:number-generator}

A \emph{binary number generator} is a function $g ∈ ℕ ⟶ S ⟶ ℕ×S$ for
some set $S$, such that for every $n$ and $s$,
$(i,s') = f(n)(s) ⟹ 0 ⩽ i < 2^{kn}$ for some $k$. A typical example is
to have $k=8$ for a generator which produces a byte at a time.

Given a binary number generator $g$ such that $g(N)$ is surjective on its first
component, our goal is to construct a general number generator $f$ as in
Section~\ref{sec:slot-leader}, such that $f(N)$ must also be surjective on its
first component.\improvement{I think it's better to just put the surjective
  requirement in the definition of (binary) random generators.}

Let us fix $N∈ℕ$ and $s₀∈S$. And define $B = \min_{n∈ℕ}N < 2^{kn}$,
and $(s_n)_{n∈ℕ}$ and $(u_n)_{n>0}$ as\improvement{I could define
  generators once and for all and these could derive from them. This
  seems to require a dependent product though. We can work around this
  defining a family of generators instead.}
\begin{itemize}
\item $s_0$ is as given above
\item $(u_{n+1}, s_{n+1}) = g(B)(s_n)$
\end{itemize}

Let $n₀ = \min_{n>0} \left(\rem{2^{kB}}{N} ⩽ u_n < 2^{kB}\right)$.

We define
$$
f(N)(s₀) = (\rem{u_{n₀}}{N}, s_{n₀})
$$

Note that, by definition of the remainder,
$\{ p∈ℕ | \rem{2^{kB}}{N} ⩽ u_n < 2^{kB} \}$ has $qN$ elements for
some $q$. From which we can deduce that $f(N)$ is surjective on its
first component (and actually: if the frequency of each $p$ in $u_n$
is the same, then the same property also holds of $f$).

\subsection{Chain quality}
\label{sec:chain-quality}

The chain qualitity function is defined as
$$
\quality{n} = 2n
$$
\subsection{Secret-sharing phases}
\label{sec:vss-phases}

The secret-sharing phases are defined as:
\begin{itemize}
\item $\commitphase = \{ n∈ℕ~|~0 ⩽ n < \quality{k} \}$
\item $\openphase = \{ n∈ℕ~|~2×\quality{k} ⩽ n < 3×\quality{k} \}$
\item $\recoveryphase = \{ n∈ℕ~|~4×\quality{k} ⩽ n < 5×\quality{k} \}$
\end{itemize}

\section{Issuing a block}
\label{sec:extending-blockchain}

\info{In this section we will specify the rules used by a slot-leader
  to create a new regular block. The state is a single chain, plus outstanding
  transactions.}

In this section we are parametrised by the current slot, a valid chain
$b₁,…,b_n$, as well as the agent id $a$ of the node attempting to issue
the block (or rather \emph{an} agent id, as it is customary in Cardano
to have many public keys for a single agent).

In order to issue a new block check that:
\begin{description}
\item[Authorised] Let $s$ be the current slot, and $l$ be the slot
  leader at $s$:
  $$
  l = \slotleader{\epochdistribution{b_1,…,b_n}}{\epochseed{b_1,…,b_n}}{\epochslot{s}}
  $$

  Then, $a$ must be transitively authorised by $l$ in that:
  $$
  i = \transitivefix{\delegations{b₁,…,b_n}}(l)
  $$
\end{description}

And build the following block:
\begin{record}
\item[predecessor] The predecessor is the (block id of) $b_n$
\item[slot number] The slot number is set to the current slot
\item[issuer] The block issuer is $a$
\item[transactions] The list of transactions is set to be the entire
  mempool\improvement{In fact this is limited by the block size
    somehow. See
    \href{https://iohk.myjetbrains.com/youtrack/issue/CDEC-170}{CDEC-170}.}
\item[delegation certificates] \improvement{Specify where the delegation
    certificates come from.}
\item[seed computation] \improvement{Specify where the ssc messages come from.}
\end{record}
\improvement{TODO: updates and where they come from}
\improvement{TODO: block size limits}

\info{Issuing a block doesn't extend the blockchain. Instead, the
  issued block is immediately received using the process of
  Section~\ref{sec:adding-block}. Issued block have the property that
  they are always valid on top of the current chain, but this
  simplifies the specification, and gives an opportunity to test the
  specification itself.}

\section{Entering the mempool}
\label{sec:entering-mempool}

This section specifies the behaviour of the mempool, which is the
space in which received transactions are stored before they are
broadcast to the network, and also where the transactions to issue a
new block (see Section~\ref{sec:extending-blockchain}) are taken
from. Many changes are considered to this section, and are discussed
in Appendix~\ref{sec:appendix-mempool}.

The mempool is only valid relatively to a given selected chain (see
Section~\ref{sec:chain-selection}). Every time the selected chain
changes (whenever a block is received), the mempool is emptied.

Let $b₁,…,b_m$ be the selected chain. The mempool is a list of
transactions $τ₁,…,τ_n$. We define a \emph{valid} mempool by induction:
\begin{itemize}
\item The empty mempool $\emptylist$ is valid
\item $τ₁,…,τ_{n+1}$ is valid if $τ₁,…,τ_n$ is valid, and $τ_{n+1}$ is
  a \emph{valid extension} of $\transactions{b₁},…,\transactions{b_m}, τ₁,…,τ_n$
\end{itemize}

A transaction $τ$ is a valid extension of a list $\overline{τ}$ (remember
that $τ$ contains not only the transaction of the mempool, but also
those of the selected chain) when
\begin{description}
  \item[Stateless]\hfill
    \begin{description}
    \item[Mempool size] The addition of the new block mustn't cause
      the mempool size to increase past its maximum value:
      $n < \maxmempoolsize$.
    \item[Valid outputs] All the outputs of the transaction must have
      amount $a > 0$.
    \item[No double-spending] The list of inputs contains no duplicate.
    \item[No redeem output] For every $out∈\outputs{τ}$,
      $\initialdistribution(\recipient{out})=0$.
    \end{description}
  \item[Stateful]\hfill
    \begin{description}
    \item[Unspent inputs] Each input of the transaction $τ$ is
      unspent. That is for each $i ∈ \inputs{τ}$
      \begin{itemize}
      \item $\intrans{i}∈\overline{τ}$ or $\intrans{i} ∈ \txinit$
      \item $\outputs{\intrans{i}}=t₁,…,t_k$, $\inindex{i} ⩽ k$
      \item for each $t ∈ \overline{τ}$, $i ∉ \inputs{t}$.
      \end{itemize}

      In these conditions, we can extend the function $\recipient{⋅}$
      and $\amount{⋅}$, as well as $\repartition{⋅}$, on the $i$ as
      \begin{itemize}
      \item $\recipient{i} = \recipient{\pos{\inindex{i}}{\outputs{\intrans{i}}}}$
      \item $\amount{i} = \amount{\pos{\inindex{i}}{\outputs{\intrans{i}}}}$
      \item $\repartition{i} = \repartition{\pos{\inindex{i}}{\outputs{\intrans{i}}}}$
      \end{itemize}
      \info{Remark: in all this rule, we use implicitly
        $\overline{τ}$, to cast $\intrans{i}∈\txids$ to an actual
        transaction. }

    \item[Conservation of value] The total amount of outputs must not exceed
      the total amount of input. A minimum fee must be paid.
      $$
      \minimumfee{τ} + ∑_{o∈\outputs{τ}} \amount{o}⩽ ∑_{i∈\inputs{τ}} \amount{i}
      $$

      If for every $i∈\inputs{τ}$,
      $\initialdistribution{\recipient{i}}>0$, that is all the inputs
      are from redeem addresses, then no fee need be paid, and the
      rules become:
      $$
      ∑_{o∈\outputs{τ}} \amount{o}⩽ ∑_{i∈\inputs{τ}} \amount{i}
      $$
    \item[Authorised] The list of witnesses has the same length as the
      list of inputs
      $$
      \length{\witnesses{τ}} = \length{\inputs{τ}}
      $$
      and for each input, the corresponding witness is the agent id of
      the input
      $$
      ∀ 0<k⩽\length{\inputs{τ}}. \recipient{\pos{k}{\inputs{τ}}} =
      \pos{k}{\witnesses{τ}}
      $$
    \end{description}

    \item[Bootstrap era] The following conditions must all pass to
    validate blocks in the bootstrap era. This specification only
    describes the behaviour during the bootstrap era, hence the
    following conditions are checked unconditionally.
    \begin{description}
    \item[Bootstrap stake delegation] For every output $o∈\outputs{τ}$, the
      stake is delegated in to the bootstrap stakeholders:
      $\repartition{o} = \bootstraprepartition$.

      \begin{treacherous}
        This is slightly imprecise because the bootstrap repartition
        has two representations in Cardano's block chain: it has a
        name (a data constructor), but it can also be represented by a
        explicit map. Only the former is actually permitted
      \end{treacherous}

    \end{description}

\end{description}

\section{Selecting the longest chain}
\label{sec:chain-selection}

Let us call a blocktree $\mathfrak{T} ∈
\blocktrees$\improvement{Clearly, a better notation for the set of
  blocktrees and the set of blockchains is necessary}{} a finite set of
valid chains, closed under prefix (\emph{i.e.} if
$b₁, …, b_{n+1} ∈ \mathfrak{T}$, then
$b₁, …, b_n ∈ \mathfrak{T}$)

\begin{implementationhint}
  The definition of blocktree is just a convenient and concise way to
  say that a blocktree is a tree of blocks ($\mathfrak{T}$ is rooted
  in the genesis state, each block in $\mathfrak{T}$ has its
  predecessor in $\mathfrak{T}$) and that the path from the root to
  any block is a valid chain.

  This would typically be implemented as a set of blocks (in the sense
  of \verb+Data.Set+). With the invariant that each block has a
  predecessor (possibly the genesis state).
\end{implementationhint}

Let us define the maximal chains
$\maxchains{\mathfrak{T}} ⊆ \mathfrak{T}$ to be the set of chains
$b₁,…,b_n$ such that for any extension $b₁,…,b_n,b ∉ \mathfrak{T}$.

\begin{implementationhint}
  Maximal chains are the equivalent, in our presentation of blocktrees
  to the leaves of the tree. In blockchain terminology, the maximal
  chains represent the tips of all the possible forks.

  Because it may be slow to retrieve the tip from a set data
  structures, it can be convenient, even in an executable
  specification, to keep a reference to these tips.
\end{implementationhint}\unsure{On the
    other hand, we don't seem to need the maximal chains in this
    specification. The only occurrence is to say that the selected
    chain is a maximal chain, but it is subsumed by the maximal-height
    condition (provided that height is monotonic). So we may be able
    to drop the maximal chain story entirely.}

A blockchain datastructure is a pair $(\mathfrak{C},\mathfrak{T}) ∈
\blockchains ⊆ \lists{\blocks}×\blocktrees$. Such that
\begin{itemize}
\item $\mathfrak{C}$ is a maximal chain of $\mathfrak{T}$:
  $$
  \mathfrak{C} ∈ \maxchains{\mathfrak{T}}
  $$
\item $\mathfrak{C}$ is of maximal height:
  $$
  \height{\mathfrak{C}} = \max_{\mathfrak{D}∈\maxchains{\mathfrak{T}}}
  \left( \height{\mathfrak{D}} \right)
  $$
\end{itemize}

We call $\mathfrak{C}$ the \emph{selected chain}. Because a typical
height function is the length of the chain, $\mathfrak{C}$ is also
often referred to as the longest chain (despite not necessarily being
unique at that height).\improvement{In the current state, the block
  chain should probably also contain the mempool, so that this section
  can describe the refreshing rules for the mempool}

Rules to add a block $b$ to the blockchain:

\begin{description}
\item[Predecessor] There must exist $b₁,…,b_n∈ \mathfrak{T}$ such that
  $b_n$ is the predecessor of $b$.
\item[Validity] $b₁,…,b_n,b$ must be a valid chain, let $\mathfrak{T}'
  = \mathfrak{T}∪\{b₁,…,b_n,b\}$
\item[Young fork] Let $b₁,…,b_i$ the longest prefix of $b₁,…,b_n$
  which is also a prefix of $\mathfrak{C}$ ($b₁,…,b_i$ is known as the
  \emph{last common ancestor} of $b₁,…,b_n$ and $\mathfrak{C}$). We
  must have $\length{\mathfrak{C}} - i ⩽ k$  (that is: the last common
  ancestor is at most $k$ block deep).
\item[Longest chain] if $\height{b₁,…,b_n,b} > \height{C}$ (note:
  larger \emph{and not equal}) then let $\mathfrak{C}'=b₁,…,b_n,b$
  else let $\mathfrak{C}'=\mathfrak{C}$.
\item[Update] The new blockchain is $(\mathfrak{C}', \mathfrak{T}')$
\end{description}

Note that this set of rules work even if $b$ had previously been
received and added to the blockchain. In this case, the
\emph{predecessor} and \emph{validity} rules will always pass (and
$\mathfrak{T}'=\mathfrak{T}$), the \emph{young fork} rule may reject
the block, in which case, by definition the blockchain stays
unchanged, the \emph{longest chain} rule will always rule that
$\mathfrak{C'}=\mathfrak{C}$. In other words, these rules are
idempotent in that receiving an already accepted block will leave the
blockchain unchanged.

\appendix

\section{Notes \& further considerations}
\label{sec:notes}

The purpose of this section is to collect information which does not
fit the specification proper, but can illuminate it, or describe how
it can be expanded to meet future requirements.

\subsection{Data validation vs validation rules}
\label{sec:data-validation-vs}

This is a general consideration for specification and implementation
design. The input received by a Cardano node is a piece of serialised
data, that is a string of bits. But validation rules for block are
done on \textsc{ast}s.

There is necessarily some validation done in the conversion from
strings of bits to \textsc{ast}: not all strings of bits decode to an
\textsc{ast}. This document refers to the rules which reject strings
of bits or make an \textsc{ast} as \emph{data validation}.

The rules to validate data which is already in \textsc{ast} form are
referred to, by contrast, as \emph{logic validation rules}.

Specification documents (this and any future document) are concerned
exclusively with logic validation rules. This raises the
question of what should be data validation rules and what should be
logic validation rules.

A an example, consider stake repartition as defined in
Section~\ref{sec:definitions}: a finite-support function $r∈\agentids
⟶ \mathbb{Q}$ such that $r(a) ⩾ 0$ and $∑_{a∈\agentids} r(a) =
1$. There are two possibilities:
\begin{itemize}
\item Take this as the definition of stake repartition in the
  \textsc{ast}
\item Take $r$ to be an arbitrary finite-support function in the
  \textsc{ast} and verify, as logic validation rules, that the two
  constraints hold
\end{itemize}
In the former case, the properties of $r$ are relegated as data
validation.

Data validation rules are invisible to specification, making
specification shorter, clearer, and easier to reason about. One may
fear that turning logic rules into validation rule may make errors
harder to spot in testing. But the contrary holds, at least in a typed
language like Haskell: data validation rules can be enforced by abstract
data types, which can be tested in isolation.

So, having as many data validation rules and as few logic validation
rules as possible makes it easier to write property-based tests, is
more composable, and make specification better. It is therefore
recommended, for future specification documents to make as many rules
as possible into data validation.

\subsection{Of forward compatibility}
\label{sec:forward-comp}

The different sections of this document have varying importance as far
as compatibility is concerned. Any change to the specification must
ensure one prime criterion: blocks which are currently stable
(\emph{i.e.} there is at least one node which has the block at depth
$k$), must be valid in the new specification.

This means, in particular, that the rules to validate incoming blocks
(Section~\ref{sec:adding-block}) must be understood scrupulously and
any evolution of the specification must make sure that any additional
criterion only affects later blocks.

At the other hand of the spectrum are the rules for the mempool and
the issuance of new blocks (Sections~\ref{sec:entering-mempool}
and~\ref{sec:extending-blockchain}) can be changed arbitrarily, as far
as compatibility is concerned: their state is completely transient,
and it won't affect how existing blocks are seen.

For example, it is, at the time of writing, the case that a node
willing to issue a test must list the entirety of transactions in this
mempool the block. As it happens, when the mempool is large, it has
proved occasionally to hard a task to perform within the 20s that
comprise a slot. It would have no effect on compatibility to allow any
sort of subset of the mempool to be selected to be part a newly issued
block.

This is also why the rules in Section~\ref{sec:entering-mempool} are
copies of rules in Section~\ref{sec:adding-block}, rather than try and
factor them into a common definition: the rules for the mempool are
likely to change quicker than those for than block validation.

\subsection{Lightweight delegation}
\label{sec:lightweight-delegation}

Lightweight delegation is an alternative to heavyweight delegation as
described with the delegation graph and delegation certificates in
this document.

In lightweight delegation, we use a cryptographic subkey signing
scheme to authorise other parties to sign in our stead. The subkey
sharing happens off chain, therefore it is harder to revoke a
certificate (since we don't benefit from the blockchain as a broadcast
channel: if we broadcast a revocation message, it can be received only
partially by the network and new blocks will issued which still accept
the subkey. The subkey is still technically valid. Plus, the economic
incentives probably don't apply, and make it costless to ignore a
revocation).

Lightweight delegation is currently implemented in Cardano, but hasn't
been deployed yet. Therefore it has not been specified in this
document yet, as this document is only concerned with the currently
used features of Cardano.

The main reason to require an off-chain delegation scheme is to be
able to keep one's private key hidden and sign for oneself using
subkeys instead. They can be made single use, at very little cost,
guaranteeing very strong security: even if a private subkey is leaked,
it is useless, and only the private key which is never used directly
is.

From the point of view of the specification, the current view is that
lightweight delegation need not appear at all: we simply assume that
the cryptography allows an agent $a$ to sign with any subkey of its
private key, whenever its private key would be allowed. In this sense,
this document is already ready for lightweight specification. But,
when time comes, we may come to the conclusion that, in fact, we need
some mechanism in the specification, to model lightweight delegation
explicitly.

\subsection{Considerations on the mempool}
\label{sec:appendix-mempool}

The checks in the mempool are expensive, and may not be the most
effective way to make the network efficient. It is being considered to
restrict the mempool to stateless tests (and correspondingly not
tracking a state of the mempool), or at least less restrictive tests.

This would allow transactions to flow throughout the Cardano network
It would allow transactions to flow throughout the Cardano network
more freely. For instance, if the network is split (in that there are
two different selected chains in the network), the two halves of the
network are not currently able to exchange transactions, making the
gossiping protocol less efficient.

Another issue arises due to race conditions. If two transactions $τ_1$ and $τ_2$
are issued, such that $τ_2$ depends on $τ_1$, and a node, due to a network
delay, receives $τ_2$ before $τ_1$, it will reject $τ_2$ instead of keeping it
in the waiting area. The current specification doesn't distinguish rejecting or
reserving blocks or transactions (maybe it will have to, eventually), but it's
also how the implementation behaves. This puts some burden on the network, and
also makes the gossiping protocol less efficient.

On the other hand this makes the work of the block issuer somewhat harder,
as they have to decide which transactions to include in their new
block from a soup of transactions rather than a structured list. It
reduces the overall computation effort (as only the block issuer must
perform the transaction ordering and stateful checks), however, it may
cause concerns for latency, as an issuer must build their block in a
20s time window. Nevertheless it may not be desirable to make all the
nodes in the network spend computation power which will be only used
by a single node (the block issuer).

A more flexible mempool implementation would keep a partial ordering
of transactions (the dependency ordering) rather than the total
topological ordering of the current mempool (as well as a waiting area
for reserved transactions which are missing an ancestor). The
restriction on the block issuer is that the transactions in a block
must be a downward-closed set of the mempool.

Independently of changing the mempool, this is the only real
restriction on the issuer. In the current mempool, where the
transactions in the mempool are totally ordered in a list, it means
taking a prefix of the list. The current system, specified in this
document, will add all the transactions from the mempool to a new
block. This is overly restrictive, and maybe too computationally
expensive to complete within a slot's time. Especially on low-power
hardware. Having low-power nodes issuing blocks is, however, a business
requirement of Cardano. Therefore this constraint should be weakened.

\subsection{Miscelaneous remarks on the current specification}
\label{sec:misc-remarks}
\begin{itemize}
\item As remarked by Manuel Chakravarty and Erik de Castro Lopo, the
  choice currently made in Cardano, that outputs be necessary
  non-empty, is dubious. There is no particular reason for it to be:
  empty output simply means that the transaction is pure
  fees. Furthermore Bruno Woltzenlogel Paleo's UTxO Chimeric Ledger
  model allows empty outputs, because it is believed to be essential
  for coloured coins.
\end{itemize}

\section{Delegation in Shelley}
\label{sec:delegation-shelley}

\newcommand{\entrepriseid}{\mathsf{dev}_\mathsf{null}}
\newcommand{\distributionslot}{\mathsf{s}_\mathsf{d}}
\newcommand{\initialpoolst}[1]{p_0(#1)}
\newcommand{\poolst}[1]{\mathsf{state}_\mathsf{pool}(#1)}

This section describes the changes to the spec required to describe
the delegation starting with the Shelley release\improvement{Some
  reference to the design document}

\begin{itemize}
\item A distinguished slot $\distributionslot ∈ ℕ$ is introduced to
  represent the first slot beyond the bootstrap era.

\item There must be a distinguished agent id to represent enterprise
  staking (\emph{i.e.} opting out of staking):
  $\entrepriseid ∈ \agentids$.

\item There must be an extra data type: a mining pool certificate is
  \begin{record}
  \item[Staking id] An agent id
  \item[Cost] a natural number
  \item[Margin] a real number in $[0,1]$
  \item[Pledge] \improvement{Exactly how this is represented is not
      settled yet. In the design spec, at time of writing, the pledge
      is expressed as a set of addresses. But it is not clear when
      this is converted to actual Adas. (this latter reflection
      belongs to the reward calculation) }
  \item[Reward id] An agent id
  \item[Destination id] An optional agent id
  \end{record}

\item Blocks must contain an additional field
  \begin{record}
  \item[Pool certificates] A list of pool certificates
  \end{record}

\item Transaction validation must be modified
  \begin{description}
  \item[Bootstrap era] These rules apply only if the slot $s$ is such
    that $s<\distributionslot$. \emph{the bootstrap era rules are
      otherwise unchanged}.
  \item[Distributed era] These rules only apply if the slot $s$ is
    such that $s > \distributionslot$.
    \begin{description}
    \item[Stake delegation] For every output $o ∈ \outputs{τ}$, the
      stake is delegated to a single agent id: the support of
      $\repartition{o}$ has cardinality $1$.
    \end{description}
  \end{description}
\item Block validation has an extra condition
  \begin{description}
  \item[Pool certificate] \improvement{TODO: validation rules for pool
      certificate. At least: no two certificate for the same staging
      id.}
  \end{description}
\item A pooling state is defined as a function
  $\agentids ⟶ ℕ×[0,1]×𝒫×\agentids×(\agentids⊎\{⋆\})$.\improvement{Here
    $𝒫$ is the set of pledges. TODO: define above, and make a macro}

  The initial pooling state function is defined as
  $$
  \initialpoolst{a} = (0,0,…,\entrepriseid,⋆)
  $$
  \unsure{I'm abusing $\entrepriseid$ here which is usually meant as a
    stake identifier, but acts here as a payment id. It would act the
    same: all rewards to $\entrepriseid$ go to waste. Question: are
    they destroyed? Or do they join a pool of unusuable Adas, which
    would still count towards total of Adas in existence?}
  Given a list of certificates $c₁,…,c_n$ of pool certificates, define
  the pooling state $\poolst{c₁,…,c_n}$:
  \begin{itemize}
  \item $\poolst{\emptylist}(a) = \initialpoolst{a}$
  \item $\poolst{c₁,…,c_{n+1}}(a)$ is
    $$
    \left\{
      \begin{array}{l@{\qquad}l}
        \poolst{c₁,…,c_{n}}(a)& \mbox{if $a$ isn't the staking id of $c_{n+1}$.}\\
        (c,m,p,r,d)& \mbox{if $c_{n+1}=(a,c,m,p,r,d)$}
      \end{array}
    \right.
    $$
  \end{itemize}
  This is generalised, the usual way, to the pool state of a list of
  blocks (cut off at last epoch).
  \begin{implementationhint}
    There is no certificate retirement in this spec. To issue a
    retirement certificate for address $a$ is the same as issuing a
    certificate with the initial value $\initialpoolst{a}$.
  \end{implementationhint}
\end{itemize}
\improvement{TODO: limit the depth of delegation. How?}

\unsure{I [aspiwack] have currently arranged the spec so that
  heavyweight delegation certificate don't count towards stake
  delegation (it didn't as far as I understand in
  bootstrap). Heavyweight certificates are used by the issuer to sign
  instead of the block leader (like lightweight certificates do). But
  it has consequences on the slot leader selection, so it's not a
  transparent choice (the issuers wouldn't be the same if heavyweight
  certificates were used to affect the stake distribution).}

\improvement{The formula for reward repartition is in the delegation
  design document. They can be copied from there when the details of
  how rewards are collected have been settled.}

\end{document}

%  LocalWords:  combinator summands repartition unallocated mempool
%  LocalWords:  stateful decrypted monoid Cardano injectivity
% LocalWords:  cardinality
